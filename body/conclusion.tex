\section{总结与展望}
本工作提出了一种面向GPU的自适应高效GNN运行时系统~\Mname。我们探索了GNN输入特征在系统优化中的潜力,并基于此设计了一系列面向GNN的系统级优化策略:

(1)二维任务调度机制和

(2)专用内存优化技术

(3)输入属性提取和轻量级参数决策模型

\noindent 以提升性能与适应性。我们在多个主流GNN模型和广泛的数据集上进行了大量实验,验证了~\Mname{}在训练与推理中的显著性能优势。总体而言,\Mname{}为用户提供了一个系统化、全面化的GPU GNN加速工具。

尽管~\Mname{}在大多数图类型和模型上表现出色,但在某些特定图结构(如II型图)上仍存在进一步优化的空间。未来的工作可以从两个方向展开:一方面,进一步引入结构感知的任务划分机制,以提升在小规模、稠密图上的性能;另一方面,为~\Mname{}添加用户定义函数(UDF)支持,以便用户可以根据特定需求自定义任务划分和内存优化策略。通过这些改进,~\Mname{}有望在更广泛的应用场景中实现更高的性能和灵活性。